\chapter{Einleitung}

%\section*{Motivation}
%\section*{Aktueller Status}
%\section*{Abgrenzungen}
%\section*{Vorgehensweise}

%Inner platform problem


Sobald eine Anwendung ein gewisses Maß an Daten persistent ablegen muss, kommt man als Entwickler nicht an Datenbanksysteme vorbei. Sie ermöglichen performante Methoden, Daten zu speichern, zu löschen und zu durchsuchen. Die bekannteste Art von Datenbanken sind relationale. Sie speichern Daten in tabellarischen Strukturen ab, zwischen denen sich Einträge gegenseitig referenzieren können. Datentypen für bestimmte Spalten ermöglichen es, Daten effizient zu speichern und unlogische Werte zu vermeiden. Das relationale Modell ermöglicht es, viele verschiedene Aufgabenstellungen in ihm abzubilden. Zusätzlich existiert mit SQL eine mächtige Sprache, die alle Aspekte der Datenverwaltung übernehmen kann. Doch nicht jede Art von Informationen kann gut von dem relationalen Modell repräsentiert werden. Besonders unter den Herausforderungen, die das Internet an moderne Anwendungen stehlt, wurden neue Datenbanken benötigt. Dazu gehören Aspekte wie Ausfallsicherheit, extreme Mengen an Benutzer, die gleichzeitig auf die Datenbank zugreifen, und Daten oder auch andere Datenmodelle, die nicht in das relationale Modell passen. In dieser Arbeit werden Lösungen für ein Problem gesucht, das nicht direkt zu den Problemen, die moderne Datenbanken lösen wollen, gehört. Viele Arten von Informationen müssen ständig erweitert werden. Dazu gehören medizinische Daten, die Elementen neue Eigenschaften zuweisen müssen, e-commerce Systeme, die zwischen allen Kategorien von Daten tausende von Attribute verwalten müssen, und Team-Kollaboration Systeme, die Benutzern ermöglichen, eigene Strukturen anzulegen und zu erweitern. Diese Systeme haben die Anforderung, das Modell dynamisch zur Laufzeit anzupassen und Objekten neue Attribute mit neuen Werten hinzuzufügen oder alte zu entfernen. Dies steht im Konflikt mit dem festen Schema von relationalen Datenbanken, das in der Regel nur bei Versionsänderungen der zugehörigen Anwendung geändert wird. Um verschiedene Ansätze, die diese Aufgabenstellung erfüllen, zu vergleichen, wurde ein Java-Programm geschrieben. Dieses Programm ermöglicht es, die Performance von typischen Anwendungsfällen, wie den CRUD-Operationen, zu vergleichen und kann beliebig mit neuen Lösungswegen erweitert werden. Der Programmcode dieses Programms ist frei unter der MIT Lizenz unter \url{https://github.com/matthias-epos/dynamic-datamodels} verfügbar. Das Programm wurde verwendet, um die Perfomance zweier Ansätze in zwei verschiedenen Datenbanksysteme zu testen und zu vergleichen.

Im Rahmen dieser Arbeit werden zunächst relevante Grundlagen relationaler Datenbanken erläutert. Im Anschluss werden die Aspekte von NO- und NewSQL Datenbanken näher betrachtet, um ihre Qualitäten und Schwächen besser einschätzen zu können. Dadurch wird ihre Tauglichkeit für die Aufgabenstellung untersucht. Im Hauptteil der Arbeit wird die Testumgebung beschrieben, mit der die relationalen Ansätze verglichen wurden. Dazu gehören die Architektur des Testprogramms, die verwendeten Technologien, Entscheidungsmerkmale und Stolpersteine, die die Entwicklung beeinflussten. Zum Abschluss werden die Ergebnisse präsentiert, statistisch untersucht und interpretiert.